\documentclass{beamer}
\beamertemplatenavigationsymbolsempty

\usetheme{Antibes}

\usepackage{fancyvrb}
\usepackage{bookmark}
\usepackage{hyperref}

\hypersetup{
    colorlinks=true,
    linkcolor=white,
    filecolor=magenta,
    urlcolor=cyan
}

\title{
Isolated and Distributed BGP Attacks, and RPKI -- From the Perspective of RouteViews
}
\author{Kevin Conte}

\newcommand{\framedgraphic}[3]{
    \begin{frame}
        \frametitle{#1}
        \begin{center}
            \includegraphics[width=\textwidth,height=0.7\textheight,keepaspectratio]{#2}
        \end{center}
        {\small Source: #3}
    \end{frame}
}

\begin{document}
\begin{frame}
    \maketitle
\end{frame}

\begin{frame}
    \frametitle{Outline}
    \begin{itemize}
        \item Background
        \item Problem and Motivation
        \item Methodology
        \item Conclusions
    \end{itemize}
\end{frame}

\begin{frame}
    \frametitle{Background}
    \pause
    {\Large BGP (Border Gateway Protocol)}
    \pause
    \begin{itemize}
        \item Protocol that allows Autonomous Systems to communicate. \pause
        \item Consists of advertisements between AS's \pause
        \item Peers advertise which prefixes they know how to get to, with the AS path to get there. \pause
        \begin{itemize}
            \item This AS path is not necessarily the shortest routing path, but it is the shortest AS path. \pause
        \end{itemize}
        \item Importantly, each advertisement includes an Origin AS. \pause
        \begin{itemize}
            \item That is, which AS is advertising that it owns a particular prefix
        \end{itemize}
    \end{itemize}
\end{frame}

\begin{frame}[fragile]
    \frametitle{BGP Announcements}

    \begin{itemize}
        \item A BGP announcement consists of the following: Timestamp, Peer ASN, Peer IP, Prefix, AS\_PATH, NEXT\_HOP, Origin AS\pause
        \item For the following example, assume the Timestamp is the same for both advertisements.\pause
        \item Also assume that the NEXT\_HOP attribute is the same as the Peer IP\pause
        \item Here, you can see that two different AS's are advertising that they own the same prefix. This is BAD.
    \end{itemize}

    \begin{Verbatim}[fontsize=\small]
Peer ASN, Peer IP, Prefix, AS_PATH, Origin AS
33437, 2001:4810::1, 2001::/32, 33437 ... 6939, 6939
3257, 2001:668:0:4::2, 2001::/32, 3257 ... 1101, 1101
    \end{Verbatim}
\end{frame}

\framedgraphic{BGP}{pictures/bgp.jpeg}{\href{http://www.noction.com/wp-content/uploads/2012/03/bgp.jpg}{noction.com}}

\begin{frame}[fragile]
    \frametitle{RPKI Overview}
    \pause
    \begin{itemize}
        \item Resource Public Key Infrastructure \pause
        \item Introduced in 2011 to add security to BGP \pause
        \item Developed by the IETF (Internet Engineering Task Force) \pause
        \item Consists of Route Origin Authorizations (ROAs)
        \begin{itemize}
            \item ASN, Prefix, Max Length, Not Before, Not After \pause
        \end{itemize}
        \item Such objects, when validated, are called Validated ROA Payloads (VRPs). \pause
        \item Example:
    \end{itemize}

    \begin{Verbatim}[fontsize=\small]
ASN,     Prefix,         Max Length, Not Before, Not After
AS12345, 128.223.0.0/16, 16,         2011-01-21, 2014-02-28
    \end{Verbatim}
\end{frame}

\begin{frame}
    \frametitle{RPKI, cont.}

    \begin{itemize}
        \item Also consists of TALs, or Trust Anchor Locations \pause
        \item RPKI is all based on trust \pause
        \item Those validating route prefixes against ROAs are trusting the TALs to provide correct information. \pause
        \item Thus, there are only a handful of TALs: \pause
        \begin{itemize}
            \item IANA (Interent Assigned Numbers Authority).
            \begin{itemize}
                \item The "root" of the Internet \pause
            \end{itemize}
            \item ARIN (American Registry for Internet Numbers) \pause
            \item APNIC (Asia-Pacific Network Information Centre) \pause
            \item AFRINIC (African Network Information Centre) \pause
            \item RIPE NCC (Réseaux IP Européens Network Coordination Centre) \pause
            \item LACNIC (Latin America and Caribbean Network Information Centre)
        \end{itemize}
    \end{itemize}
\end{frame}

\framedgraphic{RPKI}{pictures/rpki-example.jpeg}{\href{https://labs.ripe.net/Members/waehlisch/copy_of_rpkirtroverview.jpg/image_preview}{labs.ripe.net}}

\framedgraphic{RIRs}{pictures/rirs.png}{\href{https://www.ripe.net/participate/internet-governance/internet-technical-community/the-rir-system/RIPENCCServiceRegionMAP_April201402.jpg}{ripe.net}}

\begin{frame}
    \frametitle{Problem and Motivation}

    \begin{itemize}
        \item Taejoong Chung, et. al, RPKI is Coming of Age: A Longitudinal Study of RPKI Deployment and Invalid Route Origins, 2019\pause
        \item This paper shows a negative correlation between the increase in deployment of RPKI and the decrease in the number of invalid route origins.
    \end{itemize}

\end{frame}

\framedgraphic{Number of Invalid Origins}{pictures/invalid-origins-chung.png}{Chung, et. al}
\framedgraphic{RPKI Deployment across the RIR's}{pictures/rpki-deployment-chung.png}{Chung, et. al}

\begin{frame}
    \frametitle{What I Wanted To Do}

    \begin{itemize}
        \item Distinguish between BGP Hijacks and BGP Misconfigurations
    \end{itemize}

\end{frame}

\begin{frame}
    \frametitle{Why I Can't Do That}

    \begin{itemize}
        \item AIMS-KISMET 2020 -- University of California, San Diego \pause
        \begin{itemize}
            \item I had the opportunity to meet with several researchers about this topic \pause
            \item Most notably, Teejay Chung, the primary author of the aforementioned paper \pause
            \item Researchers have been attempting to do this years \pause
            \item Best tool we have is \href{https://bgpstream.com/}{CAIDA's BGPstream}
        \end{itemize}
    \end{itemize}

\end{frame}

\begin{frame}[fragile]
    \frametitle{Example of Impossibility}

    \begin{Verbatim}[fontsize=\small]
Peer AS, Peer IP, Prefix, AS PATH, Origin AS
123, 128.223.56.195, 193.56.78.0/24, 123 ... 456, 456
124, 193.57.223.16, 193.56.78.0/24, 124 ... 557, 557
    \end{Verbatim}
    \pause

    \begin{Verbatim}[fontsize=\small]
125, 190.34.56.23, 193.56.78.0/24, 125 .. 12345, 12345
    \end{Verbatim}

\end{frame}

\begin{frame}
    \frametitle{What I'm Doing}
    \pause
    \begin{itemize}
        \item Analyzing the trend of both isolated and distributed BGP attacks \pause
        \item Correlating that trend to the deployment status of RPKI \pause
        \begin{itemize}
            \item As of August 2019, RPKI now contains more than 100,000 VRPs. \pause
            \item This is promising for future success of RPKI
        \end{itemize}
    \end{itemize}
\end{frame}

\begin{frame}
    \frametitle{Datasets}
    \pause
    \begin{itemize}
        \item RouteViews
        \begin{itemize}
            \item Courtesy of the University of Oregon
            \item http://archive.routeviews.org \pause
        \end{itemize}
        \item Historical ROA data
        \begin{itemize}
            \item Courtesy of RIPE
            \item https://ftp.ripe.net/rpki
        \end{itemize}
    \end{itemize}

\end{frame}

\begin{frame}
    \frametitle{Tools Used}
    \pause
    \begin{itemize}
        \item For parsing BGP data:
        \begin{itemize}
            \item bgpreader (https://bgpstream.caida.org/docs/tools/bgpreader) \pause
        \end{itemize}
        \item For parsing Historical ROAs:
        \begin{itemize}
            \item Ziggy (https://github.com/NLnetLabs/ziggy)
            \item Routinator (https://github.com/NLnetLabs/routinator) \pause
        \end{itemize}
        \item Also, a mixture of Python 3.8+ and POSIX-compliant shell scripts
        \begin{itemize}
            \item Code to be uploaded to github soon...
        \end{itemize}
    \end{itemize}

\end{frame}

\begin{frame}
    \frametitle{Methodology}

    \begin{itemize}
        \item Define an isolated attack as two discrete AS's advertising ownership of the same prefix \pause
        \item Define a distributed attack as greater than two discrete AS's advertising ownership of the same prefix \pause
        \item Samples taken every two days from 21 January 2011 $\rightarrow$ 29 February 2020 \pause
        \item Compare the trend of isolated and distributed attacks against the deployment status of RPKI \pause
        \item Step One is to look at deployment trend of RPKI \pause
        \item Then, look at BGP Attack trends
    \end{itemize}

\end{frame}

\framedgraphic{RPKI Deployment}{pictures/rpki-deployment-self.png}{Self}

\begin{frame}
    \frametitle{A Note about the Spike}

    \begin{itemize}
        \item This is caused by APNIC migrating to a new route management system. \pause
        \item As a result, there was a bunch of incorrectly validated ROAs \pause
        \item Clearly, it was fixed quickly
    \end{itemize}

\end{frame}

\begin{frame}[fragile]
    \frametitle{Distributed Attack Example}

    \begin{itemize}
        \item Take the previouse BGP announcement example
        \item Timestamp is: 2011-01-01 12:00 +00:00
        \item Total of 7 AS's advertising ownership of the same prefix
        \item Good indicator that this is a distributed attack
    \end{itemize}

    \begin{Verbatim}[fontsize=\small]
Peer ASN, Peer IP, Prefix, AS_PATH, Origin AS
33437, 2001:4810::1, 2001::/32, 33437 ... 6939, 6939
3257, 2001:668:0:4::2, 2001::/32, 3257 ... 1101, 1101
7018, 2001:1890:111d::1, 2001::/32, 7018 ... 29259, 29259
...
    \end{Verbatim}

\end{frame}


\begin{frame}
    \frametitle{A note about the Results}

    \begin{itemize}
        \item All results presented are \emph{preliminary}
        \item Full results will be available in the report.
    \end{itemize}

\end{frame}

\begin{frame}[fragile]
    \frametitle{Results}
    \begin{table}
        \begin{tabular}{l | c | c | c }
            Internet Protocol & Prefixes & Isolated & Distrubted \\
            \hline \hline
            IPv4 & 13144978 & 273429 & 6335 \\
            IPv6 & 581418 & 6927 & 365 \\
            \hline
        \end{tabular}
        \caption{Summary}
    \end{table}
\end{frame}

\framedgraphic{IPv4 Unique Prefixes}{pictures/v4_prefixes.png}{Self}
\framedgraphic{IPv6 Unique Prefixes}{pictures/v6_prefixes.png}{Self}
\framedgraphic{IPv4 Isolated Attacks}{pictures/v4_isolated.png}{Self}
\framedgraphic{IPv6 Isolated Attacks}{pictures/v6_isolated.png}{Self}
\framedgraphic{IPv4 Distributed Attacks}{pictures/v4_distributed.png}{Self}
\framedgraphic{IPv6 Distributed Attacks}{pictures/v6_distributed.png}{Self}

\begin{frame}

    {\Large Questions?}

\end{frame}

\end{document}